%% Generated by Sphinx.
\def\sphinxdocclass{jupyterBook}
\documentclass[letterpaper,10pt,english]{jupyterBook}
\ifdefined\pdfpxdimen
   \let\sphinxpxdimen\pdfpxdimen\else\newdimen\sphinxpxdimen
\fi \sphinxpxdimen=.75bp\relax
%% turn off hyperref patch of \index as sphinx.xdy xindy module takes care of
%% suitable \hyperpage mark-up, working around hyperref-xindy incompatibility
\PassOptionsToPackage{hyperindex=false}{hyperref}

\PassOptionsToPackage{warn}{textcomp}

\catcode`^^^^00a0\active\protected\def^^^^00a0{\leavevmode\nobreak\ }
\usepackage{cmap}
\usepackage{fontspec}
\defaultfontfeatures[\rmfamily,\sffamily,\ttfamily]{}
\usepackage{amsmath,amssymb,amstext}
\usepackage{polyglossia}
\setmainlanguage{english}



\setmainfont{FreeSerif}[
  Extension      = .otf,
  UprightFont    = *,
  ItalicFont     = *Italic,
  BoldFont       = *Bold,
  BoldItalicFont = *BoldItalic
]
\setsansfont{FreeSans}[
  Extension      = .otf,
  UprightFont    = *,
  ItalicFont     = *Oblique,
  BoldFont       = *Bold,
  BoldItalicFont = *BoldOblique,
]
\setmonofont{FreeMono}[
  Extension      = .otf,
  UprightFont    = *,
  ItalicFont     = *Oblique,
  BoldFont       = *Bold,
  BoldItalicFont = *BoldOblique,
]


\usepackage[Bjarne]{fncychap}
\usepackage[,numfigreset=1,mathnumfig]{sphinx}

\fvset{fontsize=\small}
\usepackage{geometry}


% Include hyperref last.
\usepackage{hyperref}
% Fix anchor placement for figures with captions.
\usepackage{hypcap}% it must be loaded after hyperref.
% Set up styles of URL: it should be placed after hyperref.
\urlstyle{same}


\usepackage{sphinxmessages}



         \usepackage[Latin,Greek]{ucharclasses}
        \usepackage{unicode-math}
        % fixing title of the toc
        \addto\captionsenglish{\renewcommand{\contentsname}{Contents}}
        

\title{Esimerkkikirja}
\date{May 20, 2021}
\release{}
\author{Meikä Mandoliini}
\newcommand{\sphinxlogo}{\vbox{}}
\renewcommand{\releasename}{}
\makeindex
\begin{document}

\pagestyle{empty}
\sphinxmaketitle
\pagestyle{plain}
\sphinxtableofcontents
\pagestyle{normal}
\phantomsection\label{\detokenize{intro::doc}}


This is a small sample book to give you a feel for how book content is
structured.

Check out the content pages bundled with this sample book to get started.


\chapter{Content in Jupyter Book}
\label{\detokenize{content:content-in-jupyter-book}}\label{\detokenize{content::doc}}
There are many ways to write content in Jupyter Book. This short section
covers a few tips for how to do so.


\section{Markdown Files}
\label{\detokenize{markdown:markdown-files}}\label{\detokenize{markdown::doc}}
Whether you write your book’s content in Jupyter Notebooks (\sphinxcode{\sphinxupquote{.ipynb}}) or
in regular markdown files (\sphinxcode{\sphinxupquote{.md}}), you’ll write in the same flavor of markdown
called \sphinxstylestrong{MyST Markdown}.


\subsection{What is MyST?}
\label{\detokenize{markdown:what-is-myst}}
MyST stands for “Markedly Structured Text”. It
is a slight variation on a flavor of markdown called “CommonMark” markdown,
with small syntax extensions to allow you to write \sphinxstylestrong{roles} and \sphinxstylestrong{directives}
in the Sphinx ecosystem.


\subsection{What are roles and directives?}
\label{\detokenize{markdown:what-are-roles-and-directives}}
Roles and directives are two of the most powerful tools in Jupyter Book. They
are kind of like functions, but written in a markup language. They both
serve a similar purpose, but \sphinxstylestrong{roles are written in one line}, whereas
\sphinxstylestrong{directives span many lines}. They both accept different kinds of inputs,
and what they do with those inputs depends on the specific role or directive
that is being called.


\subsubsection{Using a directive}
\label{\detokenize{markdown:using-a-directive}}
At its simplest, you can insert a directive into your book’s content like so:

\begin{sphinxVerbatim}[commandchars=\\\{\}]
```\PYGZob{}mydirectivename\PYGZcb{}
My directive content
```
\end{sphinxVerbatim}

This will only work if a directive with name \sphinxcode{\sphinxupquote{mydirectivename}} already exists
(which it doesn’t). There are many pre\sphinxhyphen{}defined directives associated with
Jupyter Book. For example, to insert a note box into your content, you can
use the following directive:

\begin{sphinxVerbatim}[commandchars=\\\{\}]
```\PYGZob{}note\PYGZcb{}
Here is a note
```
\end{sphinxVerbatim}

This results in:

\begin{sphinxadmonition}{note}{Note:}
Here is a note
\end{sphinxadmonition}

In your built book.

For more information on writing directives, see the
\sphinxhref{https://myst-parser.readthedocs.io/}{MyST documentation}.


\subsubsection{Using a role}
\label{\detokenize{markdown:using-a-role}}
Roles are very similar to directives, but they are less\sphinxhyphen{}complex and written
entirely on one line. You can insert a role into your book’s content with
this pattern:

\begin{sphinxVerbatim}[commandchars=\\\{\}]
Some content \PYGZob{}rolename\PYGZcb{}`and here is my role\PYGZsq{}s content!`
\end{sphinxVerbatim}

Again, roles will only work if \sphinxcode{\sphinxupquote{rolename}} is a valid role’s name. For example,
the \sphinxcode{\sphinxupquote{doc}} role can be used to refer to another page in your book. You can
refer directly to another page by its relative path. For example, the
role syntax \sphinxcode{\sphinxupquote{\{doc\}`intro`}} will result in: {\hyperref[\detokenize{intro::doc}]{\sphinxcrossref{\DUrole{doc}{Welcome to your Jupyter Book}}}}.

For more information on writing roles, see the
\sphinxhref{https://myst-parser.readthedocs.io/}{MyST documentation}.


\subsubsection{Adding a citation}
\label{\detokenize{markdown:adding-a-citation}}
You can also cite references that are stored in a \sphinxcode{\sphinxupquote{bibtex}} file. For example,
the following syntax: \sphinxcode{\sphinxupquote{\{cite\}`holdgraf\_evidence\_2014`}} will render like
this: \sphinxcite{markdown:id3}.

Moreoever, you can insert a bibliography into your page with this syntax:
The \sphinxcode{\sphinxupquote{\{bibliography\}}} directive must be used for all the \sphinxcode{\sphinxupquote{\{cite\}}} roles to
render properly.
For example, if the references for your book are stored in \sphinxcode{\sphinxupquote{references.bib}},
then the bibliography is inserted with:

\begin{sphinxVerbatim}[commandchars=\\\{\}]
```\PYGZob{}bibliography\PYGZcb{}
```
\end{sphinxVerbatim}

Resulting in a rendered bibliography that looks like:




\subsubsection{Executing code in your markdown files}
\label{\detokenize{markdown:executing-code-in-your-markdown-files}}
If you’d like to include computational content inside these markdown files,
you can use MyST Markdown to define cells that will be executed when your
book is built. Jupyter Book uses \sphinxstyleemphasis{jupytext} to do this.

First, add Jupytext metadata to the file. For example, to add Jupytext metadata
to this markdown page, run this command:

\begin{sphinxVerbatim}[commandchars=\\\{\}]
\PYG{n}{jupyter}\PYG{o}{\PYGZhy{}}\PYG{n}{book} \PYG{n}{myst} \PYG{n}{init} \PYG{n}{markdown}\PYG{o}{.}\PYG{n}{md}
\end{sphinxVerbatim}

Once a markdown file has Jupytext metadata in it, you can add the following
directive to run the code at build time:

\begin{sphinxVerbatim}[commandchars=\\\{\}]
```\PYGZob{}code\PYGZhy{}cell\PYGZcb{}
print(\PYGZdq{}Here is some code to execute\PYGZdq{})
```
\end{sphinxVerbatim}

When your book is built, the contents of any \sphinxcode{\sphinxupquote{\{code\sphinxhyphen{}cell\}}} blocks will be
executed with your default Jupyter kernel, and their outputs will be displayed
in\sphinxhyphen{}line with the rest of your content.

For more information about executing computational content with Jupyter Book,
see \sphinxhref{https://myst-nb.readthedocs.io/}{The MyST\sphinxhyphen{}NB documentation}.


\section{Content with notebooks}
\label{\detokenize{notebooks:content-with-notebooks}}\label{\detokenize{notebooks::doc}}
You can also create content with Jupyter Notebooks. This means that you can include
code blocks and their outputs in your book.


\subsection{Markdown + notebooks}
\label{\detokenize{notebooks:markdown-notebooks}}
As it is markdown, you can embed images, HTML, etc into your posts!

\sphinxincludegraphics{{logo}.png}

You an also \(add_{math}\) and
\begin{equation*}
\begin{split}
math^{blocks}
\end{split}
\end{equation*}
or
\begin{equation*}
\begin{split}
\begin{aligned}
\mbox{mean} la_{tex} \\ \\
math blocks
\end{aligned}
\end{split}
\end{equation*}
But make sure you \$Escape \$your \$dollar signs \$you want to keep!


\subsection{MyST markdown}
\label{\detokenize{notebooks:myst-markdown}}
MyST markdown works in Jupyter Notebooks as well. For more information about MyST markdown, check
out \sphinxhref{https://jupyterbook.org/content/myst.html}{the MyST guide in Jupyter Book},
or see \sphinxhref{https://myst-parser.readthedocs.io/en/latest/}{the MyST markdown documentation}.


\subsection{Code blocks and outputs}
\label{\detokenize{notebooks:code-blocks-and-outputs}}
Jupyter Book will also embed your code blocks and output in your book.
For example, here’s some sample Matplotlib code:

\begin{sphinxVerbatim}[commandchars=\\\{\}]
\PYG{k+kn}{from} \PYG{n+nn}{matplotlib} \PYG{k+kn}{import} \PYG{n}{rcParams}\PYG{p}{,} \PYG{n}{cycler}
\PYG{k+kn}{import} \PYG{n+nn}{matplotlib}\PYG{n+nn}{.}\PYG{n+nn}{pyplot} \PYG{k}{as} \PYG{n+nn}{plt}
\PYG{k+kn}{import} \PYG{n+nn}{numpy} \PYG{k}{as} \PYG{n+nn}{np}
\PYG{n}{plt}\PYG{o}{.}\PYG{n}{ion}\PYG{p}{(}\PYG{p}{)}
\end{sphinxVerbatim}

\begin{sphinxVerbatim}[commandchars=\\\{\}]
\PYG{c+c1}{\PYGZsh{} Fixing random state for reproducibility}
\PYG{n}{np}\PYG{o}{.}\PYG{n}{random}\PYG{o}{.}\PYG{n}{seed}\PYG{p}{(}\PYG{l+m+mi}{19680801}\PYG{p}{)}

\PYG{n}{N} \PYG{o}{=} \PYG{l+m+mi}{10}
\PYG{n}{data} \PYG{o}{=} \PYG{p}{[}\PYG{n}{np}\PYG{o}{.}\PYG{n}{logspace}\PYG{p}{(}\PYG{l+m+mi}{0}\PYG{p}{,} \PYG{l+m+mi}{1}\PYG{p}{,} \PYG{l+m+mi}{100}\PYG{p}{)} \PYG{o}{+} \PYG{n}{np}\PYG{o}{.}\PYG{n}{random}\PYG{o}{.}\PYG{n}{randn}\PYG{p}{(}\PYG{l+m+mi}{100}\PYG{p}{)} \PYG{o}{+} \PYG{n}{ii} \PYG{k}{for} \PYG{n}{ii} \PYG{o+ow}{in} \PYG{n+nb}{range}\PYG{p}{(}\PYG{n}{N}\PYG{p}{)}\PYG{p}{]}
\PYG{n}{data} \PYG{o}{=} \PYG{n}{np}\PYG{o}{.}\PYG{n}{array}\PYG{p}{(}\PYG{n}{data}\PYG{p}{)}\PYG{o}{.}\PYG{n}{T}
\PYG{n}{cmap} \PYG{o}{=} \PYG{n}{plt}\PYG{o}{.}\PYG{n}{cm}\PYG{o}{.}\PYG{n}{coolwarm}
\PYG{n}{rcParams}\PYG{p}{[}\PYG{l+s+s1}{\PYGZsq{}}\PYG{l+s+s1}{axes.prop\PYGZus{}cycle}\PYG{l+s+s1}{\PYGZsq{}}\PYG{p}{]} \PYG{o}{=} \PYG{n}{cycler}\PYG{p}{(}\PYG{n}{color}\PYG{o}{=}\PYG{n}{cmap}\PYG{p}{(}\PYG{n}{np}\PYG{o}{.}\PYG{n}{linspace}\PYG{p}{(}\PYG{l+m+mi}{0}\PYG{p}{,} \PYG{l+m+mi}{1}\PYG{p}{,} \PYG{n}{N}\PYG{p}{)}\PYG{p}{)}\PYG{p}{)}


\PYG{k+kn}{from} \PYG{n+nn}{matplotlib}\PYG{n+nn}{.}\PYG{n+nn}{lines} \PYG{k+kn}{import} \PYG{n}{Line2D}
\PYG{n}{custom\PYGZus{}lines} \PYG{o}{=} \PYG{p}{[}\PYG{n}{Line2D}\PYG{p}{(}\PYG{p}{[}\PYG{l+m+mi}{0}\PYG{p}{]}\PYG{p}{,} \PYG{p}{[}\PYG{l+m+mi}{0}\PYG{p}{]}\PYG{p}{,} \PYG{n}{color}\PYG{o}{=}\PYG{n}{cmap}\PYG{p}{(}\PYG{l+m+mf}{0.}\PYG{p}{)}\PYG{p}{,} \PYG{n}{lw}\PYG{o}{=}\PYG{l+m+mi}{4}\PYG{p}{)}\PYG{p}{,}
                \PYG{n}{Line2D}\PYG{p}{(}\PYG{p}{[}\PYG{l+m+mi}{0}\PYG{p}{]}\PYG{p}{,} \PYG{p}{[}\PYG{l+m+mi}{0}\PYG{p}{]}\PYG{p}{,} \PYG{n}{color}\PYG{o}{=}\PYG{n}{cmap}\PYG{p}{(}\PYG{o}{.}\PYG{l+m+mi}{5}\PYG{p}{)}\PYG{p}{,} \PYG{n}{lw}\PYG{o}{=}\PYG{l+m+mi}{4}\PYG{p}{)}\PYG{p}{,}
                \PYG{n}{Line2D}\PYG{p}{(}\PYG{p}{[}\PYG{l+m+mi}{0}\PYG{p}{]}\PYG{p}{,} \PYG{p}{[}\PYG{l+m+mi}{0}\PYG{p}{]}\PYG{p}{,} \PYG{n}{color}\PYG{o}{=}\PYG{n}{cmap}\PYG{p}{(}\PYG{l+m+mf}{1.}\PYG{p}{)}\PYG{p}{,} \PYG{n}{lw}\PYG{o}{=}\PYG{l+m+mi}{4}\PYG{p}{)}\PYG{p}{]}

\PYG{n}{fig}\PYG{p}{,} \PYG{n}{ax} \PYG{o}{=} \PYG{n}{plt}\PYG{o}{.}\PYG{n}{subplots}\PYG{p}{(}\PYG{n}{figsize}\PYG{o}{=}\PYG{p}{(}\PYG{l+m+mi}{10}\PYG{p}{,} \PYG{l+m+mi}{5}\PYG{p}{)}\PYG{p}{)}
\PYG{n}{lines} \PYG{o}{=} \PYG{n}{ax}\PYG{o}{.}\PYG{n}{plot}\PYG{p}{(}\PYG{n}{data}\PYG{p}{)}
\PYG{n}{ax}\PYG{o}{.}\PYG{n}{legend}\PYG{p}{(}\PYG{n}{custom\PYGZus{}lines}\PYG{p}{,} \PYG{p}{[}\PYG{l+s+s1}{\PYGZsq{}}\PYG{l+s+s1}{Cold}\PYG{l+s+s1}{\PYGZsq{}}\PYG{p}{,} \PYG{l+s+s1}{\PYGZsq{}}\PYG{l+s+s1}{Medium}\PYG{l+s+s1}{\PYGZsq{}}\PYG{p}{,} \PYG{l+s+s1}{\PYGZsq{}}\PYG{l+s+s1}{Hot}\PYG{l+s+s1}{\PYGZsq{}}\PYG{p}{]}\PYG{p}{)}\PYG{p}{;}
\end{sphinxVerbatim}

There is a lot more that you can do with outputs (such as including interactive outputs)
with your book. For more information about this, see \sphinxhref{https://jupyterbook.org}{the Jupyter Book documentation}

\begin{sphinxthebibliography}{HdHPK14}
\bibitem[HdHPK14]{markdown:id3}
Christopher Ramsay Holdgraf, Wendy de Heer, Brian N. Pasley, and Robert T. Knight. Evidence for Predictive Coding in Human Auditory Cortex. In \sphinxstyleemphasis{International Conference on Cognitive Neuroscience}. Brisbane, Australia, Australia, 2014. Frontiers in Neuroscience.
\end{sphinxthebibliography}







\renewcommand{\indexname}{Index}
\printindex
\end{document}